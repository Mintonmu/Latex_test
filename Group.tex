\documentclass[a4paper, 11pt]{report}
\usepackage{blindtext}
\usepackage[T1]{fontenc}
\usepackage[utf8]{inputenc}
\usepackage{titlesec}
\usepackage{fancyhdr}
\usepackage{geometry}
\usepackage{hyperref}

\usepackage[english]{babel}
\usepackage{apacite}
\usepackage{hyperref}

\geometry{ margin=30mm }
\counterwithin{subsection}{section}
\renewcommand\thesection{\arabic{section}.}
\renewcommand\thesubsection{\thesection\arabic{subsection}.}
\usepackage{tocloft}
\renewcommand{\cftchapleader}{\cftdotfill{\cftdotsep}}
\renewcommand{\cftsecleader}{\cftdotfill{\cftdotsep}}
\setlength{\cftsecindent}{2.2em}
\setlength{\cftsubsecindent}{4.2em}
\setlength{\cftsecnumwidth}{2em}
\setlength{\cftsubsecnumwidth}{2.5em}


\begin{document}
\titleformat{\section}
{\normalfont\fontsize{15}{0}\bfseries}{\thesection}{1em}{}
\titlespacing{\section}{0cm}{0.5cm}{0.15cm}
\titleformat{\subsection}
{\normalfont\fontsize{13}{0}\bfseries}{\thesubsection}{0.5em}{}
\titlespacing{\section}{0cm}{0.5cm}{0.15cm}

%=======================================================================================

\begin{titlepage}
\center 
\textbf{\huge INFO1111: Computing 1A Professionalism}\\[0.75cm]
\textbf{\huge 2022 Semester 1}\\[2cm]
\textbf{\huge Practice: Team Project Report}\\[3cm]

\textbf{\huge Submission number: 1}\\[0.75cm]
\textbf{\huge Team Members:}\\[0.75cm]
\textbf{\large
    \begin{tabular}{|p{0.5\textwidth}|p{0.3\textwidth}|p{0.2\textwidth}|}
        \hline
        Name & Student ID & Levels being attempted in this submission\\
        \hline
        Yangqing Zheng & yzhe6302 & 1,2 \\
        Shuqing Lin & slin9149 & 1  \\
        Cathy Zhang & hzha0803 & 1,2 \\
        Shuxin Luo & shlu4802 & 1 \\        
        \hline
    \end{tabular}
}\\[0.75cm]
\end{titlepage}

%=======================================================================================

\tableofcontents

%=======================================================================================

\newpage
\section*{General Instructions}

You should use this \LaTeX\ template to generate your team project report. Keep in mind the following key points:
\begin{itemize}
    \item When we assess your report, you are not given a mark. Instead we will indicate (separately, for each team member) whether each level is ''achieved''.
    \item In order to pass the unit, you must achieve at least level 1. 
    \item In order to achieve level 2, you must first have achieved level 1, and so on for each level up to level 4. This means that we will not assess a higher level until a lower level has been achieved (though we will review one level higher and give you feedback to help you in refining your work).
    \item Some parts of the report are completed as a team and other parts require each student to complete a different section. This means that for each submission, some members of the team may have completed their work for a given section, but other members may not. It also is therefore possible that some members of the team may achieve a specified level and other members of the team may not yet have achieved that level.
    \item Even if some members are completing their material for a given level, and others are not, your team members will still need to work together to edit and compile the report.  The only exception to this is where a member of the team has already achieved the level they are targeting in a previous submission and has decided to not attempt higher levels, and so is not contributing any further (this should be obvious because no level is indicated for that student on the cover page).
    \item When completing each section you should remove the explanation text and replace it with your material.
\end{itemize}

For each submission you will add new details to this report, and/or update previous sections (where previous work was not good enough to have achieved the relevant level). In particular:

\begin{itemize}
    \item \textbf{General:} For each submission, each student can attempt up to 2 levels. You must also successfully achieve each lower level before you can be assessed at a higher level. For example, in the first submission you might attempt only level 1, but not be successful in achieving that level. You then reattempt level 1 and add in level 2 in the second submission and are successful in achieving level 1 but not level 2. For the third and final submission you could then attempt level 2, or levels 2 and 3 - or even just choose to not submit anything further and remain at level 1).
    \item \textbf{Submission 1:} You should complete at least the material for level 1 (since achieving level 1 is required to pass the unit). Each member of the team can also optionally choose to complete the material for level 2.\\
    \textit{Note 1: If you do not complete the level 2 information then you obviously cannot achieve level 2 at this stage. This does not stop you from attempting level 2 in Deliverable 2 or 3, but it will make it more difficult to achieve the higher levels later in the semester.}
    \textit{Note 2: To be able to achieve Level 1 in submission one your team has to achieve level 1 in the group component (Section 1.1) and you have to achieve Level 1 in the individual component (i.e. your assigned section 1.2, 1.3, 1.4 or 1.5)}
    \item \textbf{Submission 2:} Each member of your team will complete additional sections, but because you are submitting a single document, you need to work together to compile your results together and generate the final submission.\\
    If you did not achieve level 1 in your first submission, then you should revise the material for level 1 based on the feedback, and optionally you can also complete level 2.\\
    If you achieved level 1 in your first submission, then each team member can optionally complete the material for levels 2 and 3.
    \textit{Note: If you do not achieve level 1 with this submission then the highest level you will be able to achieve in the final submission will be level 2. If you achieve level 1, but not level 2, with this submission then the highest level you will be able to achieve with the final submission is level 3.}
    \item \textbf{Submission 3:} Again, you can correct sections where you did not achieve the specified level in the previous submission, and you complete additional sections.\\
    If you still have not achieved level 1, then you should revise the material for level 1 based on the feedback, and again optionally you can also complete level 2.\\
    For those at level 1, you can choose to complete the material for levels 2 and 3.\\
    For those at level 2, you can choose to complete the material for levels 3 and 4.\\
    For those at level 3, you can choose to complete the material for level 4.
\end{itemize}

Whilst the team project is just that -- a team project -- it has been designed to also allow different members of the team to achieve different outcomes. We do expect you to work together as a team. If you do come across problems working together then the first step should be to discuss this with your tutor. Note: If you are having problems you should approach your tutor as soon as you can to make them aware of the difficulties you are having with your team.

Finally, you should also ensure that any resources you use are suitably referenced, and references are included into the reference list at the end of this document. You should use APA 6th reference style \cite{apa6}.

%=======================================================================================

\newpage
\section{Level 1: Basic Skills}

Level 1 focuses on basic technical skills (related to \LaTeX\ and Git) and the types of skills used in different computing jobs.

\subsection{Developing industry skills}

This section is completed as a team.\\
Throughout your Computing degree we will help you learn a range of new skills. Once you graduate however you will need to continue to learn new languages, new tools, new applications, etc. For this section you need to identify 5 approaches you can take to this continual learning. You should then put these in order from most effective to least effective, and then explain the circumstances in which each approach might be appropriate. (Target = $\sim$100 words per skill = $\sim$500 words total).\\


\noindent {\large Approach 1: Access to various and updated resources on the internet \par}

\noindent Following the development of technologies, the internet has become the most efficient and effective way that people can obtain information and resources. There are abundant resources, such as tutorial lessons, solutions for solving a specific problem, and shared experiences from experts. Especially for a beginner learning a new language, online tutorials can provide you with a clear guide and detailed explanation. Those resources usually come from specialized organizations, such as the Online lectures from MIT, UC Berkeley. Accessing the top resources enables you to get a deeper understanding. Moreover, when stuck on some difficulties, the online experts comments can broaden your view and solve the problem from a new or different angle.\\

\noindent{\large Approach 2: Working collaboratively \par}

\noindent Furthermore, cooperation with each other will not only be a core trait that we are required to have, but also a helpful approach on our way to learning new languages and applications. While learning new languages, we will encounter various issues which sometimes are difficult to solve on our own after getting enough domain knowledge from online resources and thinking independently. At this time, learners who are not able to solve a specific task or struggle with designing a program, exchanging ideas and insights and seeking help are always great approaches for us. \\

\noindent{\large Approach 3: Practice, apply it in the real program. \par}

\noindent practice makes perfect. This is the most effective and efficient way to consolidate knowledge and thrust our understanding.  We can not truly understand a concept,  a function, etc unless we execute it in a container. As a learner, we cannot be too harsh on ourselves when we make mistakes, nothing is perfect at the very start of learning a new programming language, at least with more practice in programming, which in turn definitely will increase our accuracy of programming. We should take the mistakes as learning opportunities with a firm belief in the more we apply on the way to learn a programming language, the more productivity, and accuracy we will build on. \\

\noindent{\large Approach 4: Reflection or critical thinking \par}
\noindent Computer science learning activities will be substantially more efficient by critical thinking and reflective thinking. Critical thinking aids in examining the logical structure of computer learning, stimulating computer science comprehension, and addressing problems from several perspectives. Reflective thinking helps re-examine the content and results of computer science from a wide viewpoint, eradicate misunderstandings about computer science learning, and improve computer science learning efficiency. To achieve this approach, it needs critical thinking of acquiring and organizing data to uncover the answer, and then reflective thinking about the whole solution process which avoids mistakes in computer science learning, get self-improvement.\\

\noindent {\large Approach 5: Set up a clear goal and plan \par} 
\noindent Usually, learning a new language or tool involves huge work that people often don't know how to start. To be proficient and efficient for mastering a new programming language, a clear outline will guide us with clear instructions of what we are going to solve. Understanding the curriculum and breaking it down into smaller content will greatly assist person in achieving his goal. For instance, individuals should break their goal into multiple reachable step via OKD method which stands for,  objective, key results and deadline. Through this process, it increases the learning efficiency and motivation. Therefore, it is helpful to break a huge work into little pieces and set up an attainable goal each time. So people can have a clear idea about what steps they need to take in order to achieve their final goal.\\ 

\subsection{Cathy Zhang : Computer Science}
\noindent {\large 1. Pursue knowledge by self-learning \par}
\noindent As society advanced drastically, we cannot expect to learn computer science from books, so I strongly believe pursuing knowledge by self-learning is a highly demanding skill. To proficiently, efficiently and accurately access all kinds of tutorials, resources, updated information which are spread widely through either the Internet or offline method is the sustained competitive advantage. It not only saves your amounts of time to independently obtain relevant knowledge and information but also has a strong sense of innovation that tirelessly thrust yourself motivated.\\

\noindent {\large 2. Programming languages \par}
\noindent Mastering a programming language is integral to computer science, because it guarantees you to translate your ideas into a language that a computer can understand. Since the pace of technology innovation is getting faster and fiercer, writing, testing and ultimately executing programs with fundamental programming languages like Java, C, Python… will rocket you to be more competent in an organisation than other graduates. In accordance with the graduate's preferred area, whether is front-end or web development, learning various popular programming languages brings you more career perspectives and opportunities.\\

\noindent {\large 3. Strong mathematical skills \par}
\noindent It is essential to possess a comprehensive understanding of mathematical concepts. For example, mathematical statistics, calculus, linear algebra, probability, discrete mathematics … These are the fundamental theories for computer science graduates to not only develop their logic and analytical thinking but also provide the key mandatory theories for their future in-depth study of computer science, such as AI, predictive analysis and other advanced practices. Thinking about a self-driving car, how does the car decide to brake, accelerate, overtake, or even make a turn? It’s all about mathematics.\\

\noindent {\large 4. Data structure and algorithm \par}
\noindent “The more algorithms and data structures you know, the more likely you are to be able to find one that benefits your situation.”(Yang\&Crawford,  2022)I also believe it’s impossible to design efficient programs that just rely on our intuition. We usually use data structures such as arrays and matrices to solve our problems instead of using the algorithm to calculate step by step. Sometimes, we could use algorithms to accurately and comprehensively solve questions by giving clear instructions, however, we could use the data structure to get our desired output efficiently for the standardized data. Thus, the necessity of learning data structures and algorithms comes into play.\\

\noindent {\large 5. Data analysis \par}
\noindent In this information explosion era, efficiently managing the explosive growth of data becomes a critical issue. Firstly, the data structure is one of the key components of computer science in AI, operating systems etc"data structure selection can have a remarkable effect on the efficiency of the program.” (Augenstein\&Tenenbaum, 1977)In addition, analyzing data is a complementary skill for problem-solving and raising risk awareness, it also enables you to translate and visualize the numerous raw data into flowcharts, graphs, tables… At this point, data analysis comes to play whether it is future research related to data mining, database theory, or practical work such as managing large databases for companies, ensuring data security.\\

\noindent {\large 6. Master the operating systems \par}
\noindent Before we master diverse operating systems, it’s mandatory to understand the basic concepts of what is being defined as operating systems. For instance, the well-known Linux, Windows, Mac OS ... Moreover, the operating systems could be the website you’re browsing, the airline or train tickets sale systems, embedded systems in smart furniture… All these operating systems have been immersed in all the aspects of life around us, and are all need to be discussed and developed by the study of the operating system. Mastering operating systems, in another word, is mastering our lifestyle.\\

\noindent {\large 7. Software tools \par}
\noindent Without the tools, all your knowledge and theories are nothing. Proficient in using software tools such as compilers, interpreters, debuggers, application generators … is a pivotal skill. We are not only supposed to feed the operation skills and entrenched theories from books in the software tools, then implement our program. The most important thing is applying knowledge with practicing, in addition to making software tools much handier. For instance, VS Code is an application that provides hundreds of programming languages, the extensions menu provides a heap of supportive tools (markdown PDF, Python debugger…), and numerous built-in functions to help you be more productive.\\

\noindent {\large 8. Teamwork and Communication \par}
\noindent Within the fast-changing environment, a benign collaboration would maximize efficiency and productivity through minimum input of time and resources. To establish a positive team, appropriate communication skill is required to accentuate your respect, show your acknowledgment and encapsulate your ideas, which provides the optimum condition for teamwork. Then taking favor from proficiency in software tools, we can establish a collaborative environment to interact and cooperate. Asana is an example of improving team communication and cooperation. Git is another example that provides collaboration which is easier and faster to synchronize projects and swap opinions.\\

\subsection{Skills: Shuqing Lin : Data Science}
\noindent{\large 1. Data Preparation\par}
\noindent{Data preparation which includes data discovery, data transformation, data wrangling, and data cleansing is the process of preparing data for analysis. Data is critical for all forms of laster analysis processes in data science, and the accuracy of subsequent analysis and prediction models is determined by the quality of data preparation and used. As a result, data preparation will be essential for data scientists. Data scientists can be gathered from a variety of sources, processed in various data formats, and analyzed in order to comprehend the enormous volume of structure and unstructured data for effective analysis.}\\

\noindent{\large 2. Understanding and Handling Data - Data Visualization\par}
\noindent{Data science demands data scientists who can interpret the result of difficult and tedious operations into a more understandable output that can allow most people (e.g. other collaborating members) to better illustrate the meaning of data and words once the data has been collected and organized. Data visualization is the most effective method for accomplishing this purpose. Using professional tools such as tableau or ggplot, data visualization could turn a massive quantity of data into a form that is simple to comprehend and analyze such as graphs and sheets.}\\

\noindent{\large 3. Programming and Coding\par}
\noindent{Programming abilities and knowledge of a programming language may be extremely useful for data science in order to handle and comprehend large amounts of data. This skill can also benefit data scientists in furthering their knowledge and comprehension of data analysis. To boost productivity, data tools like Python, SQL are used to process massive volumes of data, manage real-time data, cloud computing, structure data, and statical parties. According to a survey by Kaggle 2020, more than 80\% of 2,675 data scientists polled utilize Python to improve their productivity which highlights the importance of this skill for data science. }\\

\noindent{\large 4.Critical Thinking \par}
\noindent{It's also significant to have critical thinking skills in order to objectively assess the facts of a  topic or situation before generating an opinion or making a decision. Thinking of data with skepticism provides a more broad and multi-faceted view of finding and analyzing problems in data science.  As well as objective analysis which aids data scientists in reaching accurate and impartial conclusions. Furthermore, owing to dependence on experience and intuition, critical thinking help data to eliminate mistakes or decision making due to reliance on intuition or experience.}\\

\noindent{\large 5.Communication Skills \par}
\noindent{In both studying and working in the industry, data science necessitates cooperation. Thus, strong communication skills are necessary for conveying knowledge more properly and effectively to others and to yourself. On the other hand, poor communication skills will lead to a lot of misunderstandings and mistakes. In order to effectively create better solutions, communication skill is required for building stronger cooperation methods, integrating and enhancing everyone’s ideas, and performing duties accurately. Being able to communicate means can be communicated with others simply and clearly.  In the working industry, excellent communication skills can assist team members in better defining their job requirements, increase collaboration effectiveness and improve team cohesiveness. }\\

\noindent{\large 6.Problem-solving Skills \par}
\noindent{The ability to deal with difficulties is important in data science since it requires addressing a wide range of challenges and finding solutions in enormous volumes of data. To address issues, a data scientist must have a brilliant mind, understand how to solve them by analyzing assumptions and resources, and apply experience to create efficient techniques for obtaining the correct answers. Being an effective data scientist requires both digging into the core cause of an issue and rapidly considering how to fix it and providing the best solution. }\\

\noindent{\large 7.Long-term Study and Intellectual Curiosity \par}
\noindent{Data science requires intellectual curiosity and inquiring mind, which is the driving force behind data scientists’ long-term learning. Long-term learning helps people who are working in related fields to learn about new materials or data trends in order to stay on top of the workplace and study. They can learn to improve their knowledge and accomplish their work or study better by reading books, internet resources, and trending for data science. Also, intellectual curiosity is the most significant and prominent aspect in the process of long-term learning, as it may lead to an exponential increase in the number of new possibilities in long-term learning.}\\

\noindent{\large 8.Statistics \par}
\noindent{Data science combines the three subjects of data engineering, mathematics, and statistics to discover a means to organize and analyze data. Therefore, statistics is one important skill for data science. It is a branch of mathematics that focuses on gathering and interpreting quantitative data using models and representations of specified data sets. It is at the core of data science, which also encompasses concepts like probability, variability, regression, and concentration patterns. Data scientists can gather, organize, analyze, interpret, and present data more effectively if they are familiar with statistical analysis, distribution curves, probabilities, standard deviations, variances, and other statistical aspects. This makes it easier for them to digest data and come up with valuable findings.}\\

\subsection{Skills: Shuxin Luo : Software Development}

\noindent 1. Data stricture and algorithm

\noindent Data structure and algorithm are the fundamental bases in many fields including computer science. Learning DSA allowing us to develop their ability to think in terms of computational language and solve problem in a more productive and complex way. In other words, it is nearly impossible to implement the process of problem-solving in a elegant way. Furthermore, “In the fast pace of technological advances, there are some constants.”\cite{DSA} DSA are core skills that many companies are seeking and demanding for due to the fast-changing industrial environment. The popular languages, tools or applications can be taken place by the more-advanced and efficient ones in one night. However, the bases like DSA will stay as the roots of the trees.\\

\noindent 2. Learn at least one language by heart

\noindent For every program developer, it is crucial to learn one language by heart.\cite{combo} When a developer is at the beginning of their career, many of them prefer to learn multiple languages at once in order to meet the enquiry of their first software development job. However, it is not a great choice since they will probably end up in a mess and distract their energy from learning other skills. Focusing and digging on one language is the best choice as it helps them to gain confidence and allows them to have more opportunities.\\

\noindent 3. Self-learning and organise

\noindent Technologies change rapidly in a fast-paced world, as a member in this field we always required to keep up-to-dated to avoid falling behind. Nowadays, self-learning as an essential skill for individuals, in order to obtain relevant knowledge and gain new insight, accessing various and updated resources on the internet such as tutorials, blog from self-learners and experience from experts is the most common method. However, individuals may also struggle to get used to the mode of self-learning. Unlike in school, we do not have specific time and syllabus to give us a clear instructions, hence setting goals and organising time become extremely important.\\

\noindent 4. Software development life cycle

\noindent Software development life cycle involves various combination of computational tasks which contributes to the well-structured process of conceiving, analysing, designing, programming, testing, deploying and maintenance. Its aim is to produce high-quality software while shorten consuming of cost and time in a systematic way. Under this methodology, lots of SDLC models and example had been investigated, such as waterfall, V-shape model and Big Bang model. While coding, it is crucial to follow the correct process and take preventive actions before the coming stage in order to produce great product.\\  

\noindent 5. Source code management\\
\noindent Managing source code is an integral part of all software development project.\cite{SDM} Source control tools such as Git and Mercurial have the most basic function including keeping history of changes made to files  and allowing multiple developers to work on the same code at the same time then merge the code that they are processing on. Without these knowledge, it will be extremely difficult for developer to collaborate as a team hence increases the reliability on each members, costs of making errors and strongly effect the efficiency of teamwork. Thus, master in control concepts such as branching and GUIs should be required skill.\cite{Git-IDE-SQL}\\

\noindent 6.IDEs\\
\noindent Integrated development environment is the most crucial tool that combines common developer tools into a single graphical user interaction. it allows developers to write, modify, compile, run and debug code. IDEs are designed to be save developer's time. For instance, developers do not need to spends hours and hours on figuring out setup process manually; the intelligent code completion also prevent developers from typing out full character. Thus, using IDEs speed up their work and maximise efficiency.\cite{IDE_intro} Furthermore, different programmers will have different choices: for C and C++ programmers, Visual Studio might be the most recommended choices.\cite{Git-IDE-SQL}\\  


\noindent 7. Testing \\
\noindent The quality of code is the top responsibility of the entire develop team, each member should has a general understanding and skill of testing. Testing is an indispensable and time consuming process to ensure software is ready for delivery to customer via preventing bugs, evaluating functions, reduce costs of development and improving performance. There are many way to test, for example: unit testing checks functions of each software units, integration testing ensures the interaction between various modules and blocks works and system testing to determine whether products reach users' expectations by an external professional testing group.\cite{SDM} Without the quality control of software through testing, a defeat may cause serious malfunctions under extreme cases, results in damaging company's reputation and leading to lost of customers. Let's consider a failure of military satellite launch which worth USD 1.2 billion due to a software bug.\\

\noindent 8. Database\\
\noindent In software development, database undertake the role of storing data of application. Developer should be familiar with both relational and documentation database, meanwhile possessing some basic database concept and skills: the ability to store records, insert, update, delete and all types of operations. SQL is the most popular common databases among developers as it allows users to access a large amount of data at once. However, it will be extremely difficult for any organisations without a database to create any kinds of applications, they will need to face many challenges including security issues and managing all records of organisations with adequate and proper backups.\cite{database_importance}\\


\subsection{Skills: Yangqing Zheng : Cyber Security}

\noindent 1. Knowledge of networking and systems administration

\noindent In general, the work of Cyber security involves defending and detecting the potential threat of the computer networks and systems from illegal access and cyber attacking. Since majority of work in cyber security is focused on networking, computer networking can be said as the backbone of internet. An in-depth comprehending of networking and systems enables you to understand the features of computers, the transmission of data and regular transactions of networks. Therefore, this background knowledge will provide you with a clear idea on how to secure the data, as well as to maintain and configure computer. \\

\noindent 2. Understanding the OSs and VMs (Operating systems and virtual machines)

\noindent To working on the field of cyber security, it is inevitably work with different businesses that using operating systems. As an cyber security expert, it is imperative to adroitly work on any operating system and study the unique characteristics of different operating systems. Therefore, this help to not be hindered by the variety of OSs. Meanwhile, Virtual machines is also an important tool that can isolated a computer environment, so allowing you to practice and research in a secure environment and consolidate your skills. Like Kali Linux, it can be said as one of most tools that cyber security required for performing penetration testing, vulnerability scanning and ethical hacking.\\

\noindent 3. Coding skill

\noindent Despite not all the works in cyber security require coding skill, it is the fundamental understanding of programming concepts and logic that can provide you with a valuable insights into how they are prone to security breaches. As the cyber security works becomes more advanced, the weight of coding knowledge and skills become more important. Once you encounter with cyber security problems that do not have the right tool, program knowledge can develop your tailored solutions to overcome it. It lets you understand tools at a deeper level, altering or composing them together. Hence it can improve your effectiveness and efficiency. \\

\noindent 4. Network security Control

Network security control basically refer to the practical skills and measures that perform in cyber security work to strength the security level of your network. so you can know how firewalls, routers and other devices work. For instance, working in cybersecurity field, it can basic to leverage a firewall to defend unauthorised traffic onto the network. Furthermore, learning the intricacies behind VPNs (virtual private networks), Intrusion detection systems, Intrusion Prevention Systems can be highly useful, which help you to maintain networks.\\

\noindent 5. Security Incident and Problem-Solving Skills 

\noindent unlike other computer work that often create things, the work of cyber security is to solve the problems that exist or prevent the hidden threats. Therefore, incident handling and problem-solving skills is imperative for cyber security workers. it is essential to know how to identity, analyse and handle imminent threats of violating organisation’s security policies. A cybersecurity professional knows how to address a complex information challenge through adopting a variety of existing and emerging technologies and digital environments. Attack such as Malware, phishing, Advanced Persistent Threats, Distributed Denial of Service are type of security breaches that often need to deal with.\\

\noindent 6. Collaboration and Communication Skills

working in any business, communication is always an important skill that allow you to work collaboratively and accurately fulfil the requirement of customers. an great communication skill enables you to clearly and concisely explain your findings, concerns, and solutions to others. For your group members, an effective communication can help you to convey technical information to them, therefore reach consensus on works and become more efficient and effective. Even for who are not professional in cyber security field, communication helps to address your idea and requirement to all levels and departments within the organisation.\\

\noindent 7. Reverse and critical thinking

\noindent In order to prevent and defend the attacks from hacking, cyber security workers often require to consider how the hacker would do and think, thus, find the solution to solve the challenge. That is called “ethical hacking’ to behave as a hacker. This ability refers to reverse and critical thinking skills, if you have no idea about how to protect, then ask yourself about how to attack. Once you have understand the skills and behaviours as a hacker, to fully understand how a system could be breached, and in turn, create effective solutions for thwarting these attacks. The more thinking you did during cyber security works, the more ability you have to effectively protect an organisation’s network and infrastructure.\\

\noindent 8. Self-learning skill

\noindent As technology advances, cybersecurity industries are also fast-changing and different attacks technologies can be developed in a predictable fast rate. the skills you learnt in college can be too regressive after going to business, therefore, who works in cyber security will required to be committed to keeping current with best practices and emerging industry trends. some way to self-learn including search resource online, reading specialised books. If there are some challenge that is too advanced without any previous incidence, it is important to use self-thinking skill. Those things can be vital for one to triumph in the field of cyber security.


%=======================================================================================

\newpage
\section{Level 2: Basic Technology}

Level 2 focuses on initial evaluation of the tech stack that is used by a selected company. All companies make use of a range of technologies, and these technologies need to work together. A tech stack is basically just this collection of technologies that collectively enable a company's systems. As an example, one of the most common technology stacks for supporting web servers is LAMP: Linux as the underlying operating system; Apache as a web server; MySQL as the supporting database; and Perl (or more recently PHP or Python) as the programming language.\\

Each student should choose a different tech stack and explain the role of each of the different technologies in that stack. Note that prior to researching your proposed tech stack and spending time writing about it, it might be a good idea to check with your tutor as to whether your chosen stack is suitable. (Target = $\sim$200-400 words per student).

\subsection{Tech Stack: MEAN: Cathy Zhang}

\noindent MEAN is a tech stack of UNIQLO, MEAN is a representative of the full-stack framework that stands for Mongo DB, Express.js, Angular JS and Node.js. It is an all-level JavaScript web stack that integrates the database, server, and frontend web page as a complete tech stack for establishing dynamic websites and applications. \\

\noindent MongoDB is a NoSQL, document-oriented database. It is different from a relational database system like MySQL, it is easy expansion, has maximum flexibility and has high efficiency for collecting and storing a huge volume of data. (Panos, 2016)In addition, it maintains data persistence due to the use of JSON document structure to enable data transmission between the layers of the database.(Rachit, 2021)\\

\noindent Express.js is a backend(server-side application framework) web framework based on the Node.js web application framework, it provides a series of solid functions for you to standardize your developed hybrid web and mobile applications. (Expressjs, n.d.)In other words, it is a minimal Node.js and has high flexibility.\\

\noindent Angular.js is a front-end (client-side application framework) web framework. It provides an easy way to add interactive functions such as user registration, product display, order management etc. The robust community support is a huge advantage of Angular.js as it is developed and maintained by Google, and can constantly receive new features and versions. (Agrawal, 2019)Moreover, Angular.js can easily build mobile and desktop applications and simply add and reuse code and components with a cross-platform framework such as Ionic. On the other hand, due to the frequent updates and releases, it is difficult to manage and maintain MEAN stack applications for beginners.\\

\noindent Node.js provides an open-source cross-platform server-side environment. It is essentially a JSinterpreter, thus, the probability of sequential and synchronous programs and the speed of the underlying JS virtual machine make it have a high performance. It takes the role of server-side execution and less dependence on the environment that doesn’t rely on any of the operating systems. In addition to this, an event-driven, non-blocking I/O model is perfect for data-intensive real-time applications to run across distributed devices, which is used by Node.js is highly efficient and lightweight.(Stackshare, n.d.)\\

\subsection{Tech Stack: MERN: Yangqing Zheng}

\noindent MERN, a popular tech-stack, is a collection of technologies that uses to develop master web applications, following the 3-tier architectural pattern, front-end display tier, application tier and database tier. Differentiating with other tech-stack, all the technologies of MERN stack are built on JavaScript language. Therefore, it becomes more convenient and friendly for software engineering to develop a full-stack web applications and dynamic websites. In recent years, MERN stack becomes increasingly popular and adopted by many global companies, for instance, Uber Technologies Inc.\\

\noindent MERN stands for MongoDB, Express, React and Node.js, which is the main four technologies that compose MERN stack.\\

\noindent MongoDB is an open-source database program that play an important role in storing data that related to your application, such as personal information, statistics etc., so it can be accessed whether it is needed. Unlike other SQL database that require a fixed schema or relational database to store data in it, MongoDB is a document-oriented database that consist of rows and columns, so it is highly flexible, allowing a variation of document structures.\\

\noindent Express.js is a web application framework for Node.js that is used for designing and building web applications easily and quickly. It allows you to structure a web application to handle multiple different HTTP requests at a specific URL and helps to respond to requests with route support so that you may write responses to specific URLs, supporting multiple templating engines to simplify generating html. There are many things can do with express.js, For instance, managing routing, sessions, user login and getting cookie value, etc.\\

\noindent React.js is a JavaScript front-end library for building user interfaces, or UIs, which is recreated by Facebook. It collects of pre-written JavaScript code so that can be reuse for common JavaScript tasks. In terms of websites and web applications, UIs are collection of components. React.js builds up on this idea of splitting your webpage into components and allows you to construct such components with JavaScript. Finally, displaying a more dynamic and aesthetic webpage.\\

\noindent Node.js is a JavaScript runtime environment that allows programmers to create different kinds of server-side applications and tools in JavaScript. It enables JavaScript code to be executed outside of a browser (such as a server), just by running scripts server-side to produce dynamic webpage content before it sends to the user’s browser. People often choose it because it is an ideal tool for building highly scalable data intensive and real time backend services that power our user applications.\\

\subsection{Tech Stack: Shuqing Lin}


\subsection{Tech Stack: add student 4 name here}

Your text goes here


%=======================================================================================

\newpage
\section{Level 3: Advanced Skills}

Level 3 focuses on more advanced technical skills (\LaTeX\ and Git) and analysis of linkages and relationships between the items in the company tech stack.

The following is a list of advanced Git and \LaTeX\ skills/features. Each student should select one pair of items from each list and demonstrate actual use of each item (either through activity in Git, or through including items in this report). (Target = $\sim$100 words per student for each feature).
\begin{itemize}
    \item Git
    \begin{itemize}
        \item Rebasing and Ignoring files
        \item Forking and Special files
        \item Resetting and Tags
        \item Reverting and Automated merges
        \item Hooks and Tags
    \end{itemize}
    \item \LaTeX\ 
    \begin{itemize}
        \item Cross-referencing and Custom commands
        \item Footnotes/margin notes and creating new environments
        \item Floating figures and editing style sheets
        \item Graphics and advanced mathematical equations
        \item Macros and hyperlinks
    \end{itemize}
\end{itemize}

\subsection{Advanced features: add student 1 name here}

Explain your use of the advanced Git and \LaTeX\ features. 

\subsection{Advanced features: add student 2 name here}

Explain your use of the advanced Git and \LaTeX\ features. 

\subsection{Advanced features: add student 3 name here}

Explain your use of the advanced Git and \LaTeX\ features. 

\subsection{Advanced features: add student 4 name here}

Explain your use of the advanced Git and \LaTeX\ features. 



%=======================================================================================

\newpage
\section{Level 4: Advanced Knowledge}

Level 4 focuses on analysing your particular tech stack and considering alternatives. Each student should consider the tech stack they described for Level 2, and then discuss each of the following points:
\begin{itemize}
    \item What are the strengths and limitations of this stack? (Target = $\sim$200 words).
    \item What alternatives exist, and under what situations might these alternatives be a better choice? (Target = $\sim$200 words).
\end{itemize}

\subsection{Advanced Knowledge: add student 1 name here}

Your text goes here

\subsection{Advanced Knowledge: add student 2 name here}

Your text goes here

\subsection{Advanced Knowledge: add student 3 name here}

Your text goes here

\subsection{Advanced Knowledge: add student 4 name here}

Your text goes here



%=======================================================================================

\newpage

\bibliographystyle{apacite}

\bibliography{main}{ Agrawal,S.  "MEAN vs. LAMP: How To Choose the Right Tech Stack - Java Code Geeks - 2022.", 2022, {\url{https://www.javacodegeeks.com/2019/09/mean-lamp-choose-the-right-tech-stack.html}}
}
\end{document}
\end{report}
